\documentclass[a4paper]{article}
\usepackage[utf8]{inputenc}
\title{Rapport de tp}
\author{Ollivier et Bergeret}
\date{\today}
\begin{document}

\maketitle

\newpage

\section{Introduction}

Dans ce TP, nous allons introduire la programmation Assembleur.

\section{Reponses}
\subsection{Q1} 
Nous savons que la valeur maximale pouvant être prise par suite de nbits est $2^{n}-1$. Par conséquent:\\
eax = $2^{32}-1$\\
ax = $2^{16}-1$\\
ah = al = $2^{8}-1$\\

\subsection{Q2}
l'instruction mov permet d'affecter la valeur de l'opérande de droite vers l'opérande de gauche.\\

Nous convertissons tout d'abord le nombre en base décimale vers une base hexadécimale:\\
134512768 = 0x8048080\\
Nous savons maintenant que eax contient cette valeur:\\
eax = 0x8048080\\
Par conséquent, en exploitant les sous registres d'eax, nous pouvons accèder à des portions de valeurs différentes.\\
ax = 0x8080\\
ah = 0X80\\
al = 0x80\\

\subsection{Q3}
L'instruction add permet d'ajouter la valeur contenue dans l'opérande de droite vers l'opérande de gauche. Le résultat est sauvegardé dans l'opérande de gauche\\
eax = 0xF000E0F0\\
ax = 0xE0F0\\
ah = 0xE0\\
al = 0xF0\\

\subsection{Q4}
Afin d'obtenir la capacité de stockage totale, nous prenons la valeur de $2^32$, soit\\ %TODO insérer la valeur de 2^32
3,73 Go (arrondi à $10^-2$)

\subsection{Q5}
Quand entre crochets, le nombre fait référence à la valeur située à l'adresse donnée. Sinon, il s'agit de la valeur elle-même

\subsection{Q6}
La case
ax = 30

\subsection{Q7}
eax = OxA502

\subsection{Q8}
0: 0x01\\
1: 0x02\\
2: 0x01\\
3: 0x01\\

\subsection{Q9}
0: 0x01\\
1: 0x21\\
2: 0x01\\
3: 0x00\\

\subsection{Q10}
eax (little endian): 0x01\\
eax (big endian):0x00

\section{Conclusion}
\end{document}
